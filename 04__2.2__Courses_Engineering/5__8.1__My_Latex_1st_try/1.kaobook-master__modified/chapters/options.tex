\setchapterpreamble[u]{\margintoc}
\chapter{类选项}
\labch{options}

在本章中,我将描述最常用的选项,从\Class{scrbook}继承的选项和特定于\Class{kao}的选项。传递给类的选项修改其默认行为;不过要注意,有些选项可能会导致意想不到的结果\ldots

\section{\Class{KOMA} options}

\Class{kaobook}类基于\Class{scrbook},因此它理解您通常传递给该类的所有选项。如果您有足够的耐心,可以阅读\KOMAScript\xspace 指南。\sidenote{指南可以从\url{https://ctan.org/pkg/koma-script?lang=en}.}下载。

加载类时传递给该类的每个\KOMAScript\xspace 选项都会自动激活。此外,在\Class{kaobook}中,一些选项修改了默认值。例如,字体大小为9.5pt,段落之间用空格分隔,\sidenote[-7mm][]{精确地说,段落之间用半行空格分隔:\Option{parskip}的值是\enquote{half}。}没有缩进标记。

\section{\Class{kao} options}

In the future I plan to add more options to set the paragraph formatting 
(justified or ragged) and the position of the margins (inner or outer in 
twoside mode, left or right in oneside mode).\sidenote{As of now, 
paragraphs are justified, formatted with \Command{singlespacing} (from 
the \Package{setspace} package) and \Command{frenchspacing}.}

I take this opportunity to renew the call for help: everyone is 
encouraged to add features or reimplement existing ones, and to send me 
the results. You can find the GitHub repository at 
\url{https://github.com/fmarotta/kaobook}.

\begin{kaobox}[frametitle=To Do]
实现\Option{justify}和\Option{margin}选项。为了与\KOMAScript\xspace 样式保持一致,它们应该接受一个简单开关作为参数,其中简单开关应该是\Option{true}或\Option{false},或者\KOMAScript 支持的简单开关的其他标准值之一。有关更多信息,请参阅\KOMAScript\xspace 文档。
\end{kaobox}

The above box is an example of a \Environment{kaobox}, which will be 
discussed more thoroughly in \frefch{mathematics}. Throughout the book I 
shall use these boxes to remarks what still needs to be done.

\section{Other things worth knowing}

A bunch of packages are already loaded in the class because they are 
needed for the implementation. These include:

\begin{itemize}
	\item etoolbox
	\item calc
	\item xifthen
	\item xkeyval
	\item xparse
	\item xstring
\end{itemize}

Many more packages are loaded, but they will be discussed in due time. 
Here, we will mention only one more set of packages, needed to change 
the paragraph formatting (recall that in the future there will be 
options to change this). In particular, the packages we load are:

\begin{itemize}
	\item ragged2e
	\item setspace
	\item hyphenat
	\item microtype
	\item needspace
	\item xspace
	\item xcolor (with options \Option{usenames,dvipsnames})
\end{itemize}

Some of the above packages do not concern paragraph formatting, but we 
nevertheless grouped them with the others. By default, the main text is 
justified and formatted with singlespacing and frenchspacing; the margin 
text is the same, except that the font is a bit smaller.

\section{Document Structure}

We provide optional arguments to the \Command{title} and 
\Command{author} commands so that you can insert short, plain text 
versions of this fields, which can be used, typically in the half-title 
or somewhere else in the front matter, through the commands 
\Command{@plaintitle} and \Command{@plainauthor}, respectively. The PDF 
properties \Option{pdftitle} and \Option{pdfauthor} are automatically 
set by hyperref to the plain values if present, otherwise to the normal 
values.\sidenote[-1.4cm][]{We think that this is an important point so 
we remark it here. If you compile the document with pdflatex, the PDF 
metadata will be altered so that they match the plain title and author 
you have specified; if you did not specify them, the metadata will be 
set to the normal title and author.}

There are defined two page layouts, \Option{margin} and \Option{wide}, 
and two page styles, \Option{plain} and \Option{fancy}. The layout 
basically concern the width of the margins, while the style refers to 
headers and footer; these issues will be 
discussed in \frefch{layout}.\sidenote{For now, suffice it to say that pages with 
the \Option{margin} layout have wide margins, while with the 
\Option{wide} layout the margins are absent. In \Option{plain} pages the 
headers and footer are suppressed, while in \Option{fancy} pages there 
is a header.} 

The commands \Command{frontmatter}, \Command{mainmatter}, and 
\Command{backmatter} have been redefined in order to automatically 
change page layout and style for these sections of the book. The front 
matter uses the \Option{margin} layout and the \Option{plain} page 
style. In the mainmatter the margins are wide and the headings are 
fancy. In the appendix the style and the layout do not change; however 
we use \Command{bookmarksetup\{startatroot\}} so that the bookmarks of 
the chapters are on the root level (without this, they would be under 
the preceding part). In the backmatter the margins shrink again and we 
also reset the bookmarks root.
