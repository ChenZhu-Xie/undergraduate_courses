\setchapterstyle{kao}
\setchapterpreamble[u]{\margintoc}
\chapter{数学及盒子}
\labch{mathematics}

\section{定理}

尽管大多数人抱怨看到一本充满公式式的书,数学却是许多书的重要组成部分。在这里,我们将说明一些可能性。我们认为定理、定义、注释和例子都应该在阴影的背景下加以强调;然而,颜色不应该是沉重的眼睛,所以我们选择了一种淡黄色。\sidenote{这里的所有框都是相同的颜色,因为我们不希望我们的文档看起来像\href{https://en.wikipedia.org/wiki/Harlequin}{Harlequin}。}

\begin{definition}
\labdef{openset}
Let $(X, d)$ be a metric space. A subset $U \subset X$ is an open set 
if, for any $x \in U$ there exists $r > 0$ such that $B(x, r) \subset 
U$. We call the topology associated to d the set $\tau\textsubscript{d}$ 
of all the open subsets of $(X, d).$
\end{definition}

\refdef{openset} 是非常重要的。我不是在开玩笑,但是我插入这个短语只是为了说明如何引用定义。下面的语句在不同的环境中反复出现。

\begin{theorem}
A finite intersection of open sets of (X, d) is an open set of (X, d), 
i.e $\tau\textsubscript{d}$ is closed under finite intersections. Any 
union of open sets of (X, d) is an open set of (X, d).
\end{theorem}

\begin{proposition}
A finite intersection of open sets of (X, d) is an open set of (X, d), 
i.e $\tau\textsubscript{d}$ is closed under finite intersections. Any 
union of open sets of (X, d) is an open set of (X, d).
\end{proposition}

\marginnote{You can even insert footnotes inside the theorem 
	environments; they will be displayed at the bottom of the box.}

\begin{lemma}
A finite intersection\footnote{I'm a footnote} of open sets of (X, d) is 
an open set of (X, d), i.e $\tau\textsubscript{d}$ is closed under 
finite intersections. Any union of open sets of (X, d) is an open set of 
(X, d).
\end{lemma}

您可以安全地忽略定理\ldots 的内容,我假设,如果您对课本中的定理感兴趣,那么您已经了解了一些关于添加它们的经典方法。这些示例应该只显示您在这个类中可以做的所有事情。

\begin{corollary}[Finite Intersection, Countable Union]
A finite intersection of open sets of (X, d) is an open set of (X, d), 
i.e $\tau\textsubscript{d}$ is closed under finite intersections. Any 
union of open sets of (X, d) is an open set of (X, d).
\end{corollary}

\begin{proof}
证明留给读者作为一个简单的练习。提示: \zhlipsum[2]
\end{proof}

\begin{definition}
Let $(X, d)$ be a metric space. A subset $U \subset X$ is an open set 
if, for any $x \in U$ there exists $r > 0$ such that $B(x, r) \subset 
U$. We call the topology associated to d the set $\tau\textsubscript{d}$ 
of all the open subsets of $(X, d).$
\end{definition}

\marginnote{
	Here is a random equation, just because we can:
	\begin{equation*}
  x = a_0 + \cfrac{1}{a_1
          + \cfrac{1}{a_2
          + \cfrac{1}{a_3 + \cfrac{1}{a_4} } } }
	\end{equation*}
}

\begin{example}
Let $(X, d)$ be a metric space. A subset $U \subset X$ is an open set 
if, for any $x \in U$ there exists $r > 0$ such that $B(x, r) \subset 
U$. We call the topology associated to d the set $\tau\textsubscript{d}$ 
of all the open subsets of $(X, d).$
\end{example}

\begin{remark}
Let $(X, d)$ be a metric space. A subset $U \subset X$ is an open set 
if, for any $x \in U$ there exists $r > 0$ such that $B(x, r) \subset 
U$. We call the topology associated to d the set $\tau\textsubscript{d}$ 
of all the open subsets of $(X, d).$
\end{remark}

As you may have noticed, definitions, example and remarks have 
independent counters; theorems, propositions, lemmas and corollaries 
share the same counter.

\begin{remark}
Here is how an integral looks like inline: $\int_{a}^{b} x^2 dx$, and 
here is the same integral displayed in its own paragraph:
\[\int_{a}^{b} x^2 dx\]
\end{remark}

We provide two files for the theorem styles: 
\href{style/plaintheorems.sty}{plaintheorems.sty}, which you should 
include if you do not want coloured boxes around theorems; and 
\href{style/mdftheorems.sty}{mdftheorems.sty}, which is the one used for 
this document.\sidenote{The plain one is not showed, but actually it is 
exactly the same as this one, only without the yellow boxes.} Of course, 
you will have to edit these files according to your taste and the 
general style of the book.

\section[Boxes \& Environments]{Boxes \& Custom Environments
\sidenote[*1.6][]{Notice that in the table of contents and in the 
	header, the name of this section is \enquote{Boxes \& Environments}; 
	we achieved this with the optional argument of the \texttt{section} 
	command.}}

Say you want to insert a special section, an optional content or just 
something you want to emphasise. We think that nothing works better than 
a box in these cases. We used \Package{mdframed} to construct the ones 
shown below. You can create and modify such environments by editing the 
provided file \href{style/environments.sty}{environments.sty}.

\begin{kaobox}[frametitle=盒子标题]
\zhlipsum[3]
\end{kaobox}

如果设置了计数器,甚至可以创建自己的编号环境。

\begin{kaocounter}
\zhlipsum[4]
\end{kaocounter}

\section{Experiments}

也可以在盒子里包装边注。我们鼓励大胆的读者尝试自己的实验,并让我知道结果。

\marginnote[-2.2cm]{
	\begin{kaobox}[frametitle=title of margin note]
		使用kaobox盒子的边注.\\
		(实际上, kaobox是在marginnote里面!)
	\end{kaobox}
}

我相信许多其他特殊的事情是可能的与类\Class{kaobook}类。在开发过程中,我努力使它尽可能灵活,这样就可以不费太大力气地添加新特性。因此,我希望你们能在这门课的写作中找到最好的方式来表达自己,写一本书,写一篇报告或者写一篇论文,我也很想看看你们可以尝试的任何实验的结果。

%\begin{margintable}
	%\captionsetup{type=table,position=above}
	%\begin{kaobox}
		%\caption{caption}
		%\begin{tabular}{ |c|c|c|c| }
			%\hline
			%col1 & col2 & col3 \\
			%\hline
			%\multirow{3}{4em}{Multiple row} & cell2 & cell3 \\ & cell5 
			%%& cell6 \\ 
			%& cell8 & cell9 \\
			%\hline
		%\end{tabular}
	%\end{kaobox}
%\end{margintable}
